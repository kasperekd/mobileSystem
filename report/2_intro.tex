\chapter*{Введение}
\addcontentsline{toc}{chapter}{Введение}
\label{ch:intro}

Развитие современных систем связи требует постоянного совершенствования методов передачи информации для обеспечения высокой надежности, эффективности и устойчивости к помехам. Особую роль играют технологии цифровой обработки сигналов, позволяющие реализовать сложные алгоритмы кодирования, модуляции, эквалайзирования и демодуляции.

В рамках данного цикла практических работ проводится комплексное исследование и реализация основных функциональных блоков системы цифровой связи с использованием программной среды MATLAB. Целью работы является изучение принципов построения современных систем связи, а также получение практических навыков по реализации ключевых алгоритмов обработки сигналов.

В ходе выполнения практических работ будут последовательно реализованы и исследованы следующие компоненты: знаковое кодирование и декодирование, помехоустойчивое кодирование (сверточное кодирование и декодирование Витерби), перемежение битовой последовательности, QPSK модуляция, OFDM модуляция, модель канала передачи с замираниями и аддитивным белым гауссовским шумом, а также приемная часть системы, включающая эквалайзирование и OFDM демодуляцию. Завершающий этап работы посвящен анализу производительности реализованной системы путем расчета коэффициента битовых ошибок (BER) и построения графиков.

Каждая практическая работа включает в себя теоретическое описание реализуемого блока, детализацию его программной реализации на языке MATLAB, демонстрацию результатов тестирования в виде консольного вывода и, при необходимости, графических представлений.

Полученные в результате работы знания и навыки будут способствовать пониманию принципов функционирования современных цифровых систем связи и станут основой для дальнейшего изучения и разработки более сложных телекоммуникационных систем.

\endinput