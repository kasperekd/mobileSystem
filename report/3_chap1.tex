\chapter{Знаковое кодирование}

\section{Введение}

Знаковое кодирование представляет собой метод представления текстовой информации в двоичном формате. Основная цель этой практической работы – реализовать процесс кодирования и декодирования текстового сообщения, используя заданный алфавит и разрядность кодового слова.

\section{Описание знакового кодера}

Знаковый кодер предназначен для преобразования входного символьного сообщения в последовательность двоичных символов (бит). Исходное сообщение состоит из латинских букв (маленьких и заглавных), цифр (0–9), пробела и точки, всего 64 символа.

\subsection{Принцип работы кодера}

Каждому символу входного сообщения присваивается уникальный двоичный код.

Количество бит для кодирования одного символа определяется разрядностью кодового слова.

Итоговое битовое сообщение формируется путем последовательного соединения кодов всех символов исходного текста.

\subsection{Описание входных и выходных данных}

Входные данные: текстовое сообщение длиной от 30 до 100 символов.

Выходные данные: битовая последовательность, представляющая закодированное сообщение.

\section{Описание знакового декодера}

Знаковый декодер выполняет обратную операцию: преобразует битовую последовательность обратно в текстовое сообщение. Для этого используется та же знаковая кодировка, что и в кодере.

\subsection{Принцип работы декодера}

Битовое сообщение разделяется на фрагменты, соответствующие длине кодового слова.

Каждому фрагменту сопоставляется соответствующий символ из заданного алфавита.

Восстанавливается исходное текстовое сообщение.

\subsection{Описание входных и выходных данных}

Входные данные: битовая последовательность.

Выходные данные: исходное текстовое сообщение.

\section{Выводы}

В данной работе был рассмотрен процесс знакового кодирования и декодирования текстового сообщения. Реализованный кодер позволяет эффективно преобразовывать текст в битовую форму, а декодер успешно восстанавливает исходные данные. Этот метод широко применяется в системах обработки и хранения информации.